%% report.tex
%%
%% TP MLEA - d-hall_f

\documentclass[xcolor=dvipsnames]{beamer}

\usepackage{beamerthemesplit}
\usetheme{Warsaw}


\mode<presentation>
\usecolortheme{orchid}
%\setbeamertemplate{navigation symbols}{} % vire la nav'barre

\mode<presentation>
% Change the default bottom line to include the page number.
\defbeamertemplate*{footline}{infolines theme without institution}
{
  \leavevmode%
  \hbox{%
    \begin{beamercolorbox}[wd=.25\paperwidth,ht=2.25ex,dp=1.125ex,center]{author in head/foot}%
      \usebeamerfont{author in head/foot}\insertshortauthor
    \end{beamercolorbox}%
    \begin{beamercolorbox}[wd=.50\paperwidth,ht=2.25ex,dp=1.125ex,center]{title in head/foot}%
      \usebeamerfont{title in head/foot}CSI Seminar 2009 - \insertshorttitle
    \end{beamercolorbox}%
    \begin{beamercolorbox}[wd=.25\paperwidth,ht=2.25ex,dp=1.125ex,right]{date in head/foot}%
      \usebeamerfont{date in head/foot}%
      % \insertshortdate{}\hspace*{2em}
      \insertframenumber{} / \inserttotalframenumber\hspace*{2ex}
    \end{beamercolorbox}}%
  \vskip0pt%
}
\mode
<all>

% Colors by Yabo :)
\definecolor{dark-blue}{rgb}{0.254,0.274,0.545}
\definecolor{vlblue}{rgb}{0.800,0.880,0.958}
\definecolor{light-blue}{rgb}{0.670,0.780,0.898}
\definecolor{mid-blue}{rgb}{0.458,0.521,0.717}
\definecolor{lred}{rgb}{1,0.219,0.219}

\definecolor{euyellow}{rgb}{0.8,0.75,0.1} % Definition of yellow color
\definecolor{eublue}{rgb}{0.1,0.1,0.5} % Definition of blue color
\definecolor{olrde-green}{rgb}{0.02,0.58,0.34}
\definecolor{lrde-green}{rgb}{0.34,0.82,0.61}
\definecolor{blrde-green}{rgb}{0.11, 0.32, 0.25}
\definecolor{lgreen}{rgb}{0.68, 0.98, 0.78}
\definecolor{black}{rgb}{0,0,0}

\setbeamercolor{structure}{fg=dark-blue}

% Main title-bar colors
\setbeamercolor{palette primary}{fg=white,bg=mid-blue!100} % !<#> is transparency
%\setbeamercolor{palette secondary}{fg=red!88,bg=green!20}
\setbeamercolor{palette secondary}{fg=dark-blue,bg=light-blue}
\setbeamercolor{palette tertiary}{fg=white,bg=eublue!90}
\setbeamercolor{palette quaternary}{fg=white,bg=dark-blue}

%\setbeamercolor{titlelike}{parent=palette quaternary}

% Inner block colors
\setbeamercolor{block title}{fg=eublue,bg=light-blue}


\setbeamercolor{example text}{fg=eublue,bg=light-blue}

\setbeamercolor{block title alerted}{use=alerted text,fg=eublue,bg=alerted text.fg!75!bg}
\setbeamercolor{block title example}{use=example text,fg=white,bg=example text.fg!75!bg}

%%\setbeamercolor{block body}{parent=normal text,use=block title,bg=block title.bg!25!bg}
\setbeamercolor{block body}{bg=light-blue!30!}

\setbeamercolor{block body alerted}{parent=normal text,use=block title alerted,bg=block title alerted.bg!25!bg}
\setbeamercolor{block body example}{parent=normal text,use=block title example,bg=block title example.bg!25!bg}

\setbeamercolor{sidebar}{bg=euyellow!70}

\setbeamercolor{palette sidebar primary}{fg=eublue}
\setbeamercolor{palette sidebar secondary}{fg=eublue!75}
\setbeamercolor{palette sidebar tertiary}{fg=eublue!75}
\setbeamercolor{palette sidebar quaternary}{fg=red}

\setbeamercolor*{separation line}{}
\setbeamercolor*{fine separation line}{}

\setbeamercovered{dynamic}

\mode
<all>


\usepackage[english]{babel}
\usepackage[latin1]{inputenc}
\usepackage{times}
\usepackage[T1]{fontenc}
\usepackage{graphicx}
\usepackage{listings}
\usepackage{floatflt}
\usepackage{wrapfig}

%\usepackage[dvips]{graphicx,color}
%\usepackage[pdftex,colorlinks]{hyperref}
%\usepackage{hyperref}

\title{MLEA - TP 2}
\author{Florent D'Halluin}
\date{8 Juillet 2009}
\institute{EPITA 2010}

%%\AtBeginSubsection[]
%%{
%%  \begin{frame}<beamer>
%%    \frametitle{Outline}
%%    \tableofcontents[currentsection,currentsubsection]
%%  \end{frame}
%%}

\begin{document}

\begin{frame}
  \titlepage
\end{frame}

\begin{frame}
  \frametitle{Plan}
  \tableofcontents
\end{frame}

\section{Classification}
\subsection{Estimer l'efficacit� d'un classifieur}
\subsubsection{K-Fold cross-validation}
\begin{frame}
  \frametitle{K-Fold cross-validation}
  \begin{itemize}
    \item{Divise les donn�es en $k$ sections}
    \item{$k$ it�rations}
    \item{$k-1$ sections d'apprentissage}
    \item{$1$ section de test}
    \item{R�sultat: moyenne des taux de reconnaissances et �cart-type}
    \item{Ici, $k = 10$}
  \end{itemize}
\end{frame}

\subsubsection{Courbe ROC}
\begin{frame}
  \frametitle{Courbe ROC}
  \begin{itemize}
    \item{Se base sur le taux de certitude de chaque point reconnu}
    \item{Fonction du taux de vrais positifs et de faux positifs}
  \end{itemize}
  \begin{figure}[ht]
    \begin{center}
      \includegraphics[width=60mm]{../classification/donut_data_knns_0_roc}
      \caption{Courbe ROC pour plusieurs classifieurs, sur donut.}
      \label{fig:roc}
    \end{center}
  \end{figure}
\end{frame}

\subsection{Classification de donn�es continues}
\subsubsection{KNN}
\begin{frame}
  \frametitle{KNN}
  \begin{itemize}
    \item{On peut calculer la distance entre deux points du dataset}
    \item{Trouver les K plus proches voisins}
  \end{itemize}
\end{frame}

\subsubsection{Normal distribution}
\begin{frame}
  \frametitle{Normal distribution}
  \begin{itemize}
  \item{Mod�liser les donn�es d'apprentissage par une distribution
    normale (moyenne, �cart-type)}
    \item{Maximiser la probabilit� � post�riori}
  \end{itemize}
\end{frame}

\begin{frame}
  \frametitle{Comparaison}
  \begin{figure}[ht]
    \begin{center}
      \includegraphics[width=75mm]{../classification/optdigits_data_tra_pdf_200_kfcrossval}
      \caption{Comparaison de classifieurs sur une partie du dataset optdigits
        (training).}
      \label{fig:pdf_optd_200}
    \end{center}
  \end{figure}
\end{frame}

\subsection{Classification de donn�es discr�tes}
\subsubsection{Na�ve Bayes classifier}
\begin{frame}
  \frametitle{Na�ve Bayes classifier}
  \begin{itemize}
    \item{Donn�es � caract�ristiques ind�pendantes}
    \item{Estimer certaines probabilit�s � partir des donn�es
      d'apprentissage ($P(C = c) et P(Xi = xi | C)$)}
    \item{Maximiser $P(C = c|X = x)$}
  \end{itemize}
\end{frame}

\subsubsection{Continuousification}
\begin{frame}
  \frametitle{Continuousification}
  \begin{itemize}
    \item{Rendre des donn�es discr�tes continues (pour utiliser KNN)}
    \item{Conserver la d�pendance entre les variables}
    \item{Na�f: Une valeur arbitraire par valeur observ�e}
    \item{NBF: Une dimension par valeur observ�e}
    \item{VDM/MDV: Estimer et utiliser les probabilit� conditionnelles}
  \end{itemize}
\end{frame}

\begin{frame}
  \frametitle{Comparaison}
  \begin{figure}[ht]
    \begin{center}
      \includegraphics[width=75mm]{../classification/mushroom_data_cnt_100_kfcrossval}
      \caption{Comparaison de classifieurs sur une partie du dataset mushroom.}
      \label{fig:mushroom_data_cnt_100_kfcrossval}
    \end{center}
  \end{figure}
\end{frame}

\section{Clustering}
\subsection{Estimer l'efficacit� du clustering}
\begin{frame}
  \frametitle{Estimer l'efficacit� du clustering}
  \begin{itemize}
    \item{D�tecter la convergence vers un extr�mum local}
    \item{Somme des distances intra-cluster}
    \item{Vitesse de convergence: nombre d'it�rations}
  \end{itemize}
\end{frame}

\subsection{M�thodes de clustering}
\subsubsection{K-MEANS}
\begin{frame}
  \frametitle{K-MEANS}
  \begin{itemize}
    \item{Choisir $k$ centres au hasard}
    \item{Calculer les clusters et les nouveaux centres}
    \item{It�rer jusqu'� la convergence}
  \end{itemize}
\end{frame}

\begin{frame}
  \frametitle{K-MEANS - centres initiaux}
  \begin{figure}[ht]
    \begin{center}
      \includegraphics[width=75mm]{../clustering/teddy_toy_data_0_clustering_K-MEANS(16)_Start}
      \caption{K-MEANS clustering sur le dataset teddy-toy, centres initiaux.}
      \label{fig:teddy_toy_data_02}
    \end{center}
  \end{figure}
\end{frame}

\subsubsection{Distance maximale}
\begin{frame}
  \frametitle{Distance maximale}
  \begin{itemize}
    \item{Optimiser le choix des centres initiaux}
    \item{Maximiser la distance avec les centres d�j� choisis}
    \item{Sensible au bruit}
  \end{itemize}
\end{frame}

\begin{frame}
  \frametitle{Distance maximale - centres initiaux}
  \begin{figure}[ht]
    \begin{center}
      \includegraphics[width=75mm]{../clustering/teddy_toy_data_0_clustering_K-MEANSPP(16)_(max_distance)_Start}
      \caption{K-MEANSPP (max distance) clustering sur le dataset teddy-toy, centres initiaux.}
      \label{fig:teddy_toy_data_06}
    \end{center}
  \end{figure}
\end{frame}

\subsubsection{K-MEANS++}
\begin{frame}
  \frametitle{K-MEANS++}
  \begin{itemize}
    \item{Optimiser le choix des centres initiaux}
    \item{Choisir en fonction de la distance aux centres d�j� choisis}
    \item{R�sistant au bruit}
    \item{Efficace}
  \end{itemize}
\end{frame}

\begin{frame}
  \frametitle{K-MEANS++ - centres initiaux}
  \begin{figure}[ht]
    \begin{center}
      \includegraphics[width=75mm]{../clustering/teddy_toy_data_0_clustering_K-MEANSPP(16)_Start}
      \caption{K-MEANSPP clustering sur le dataset teddy-toy, centres initiaux.}
      \label{fig:teddy_toy_data_03}
    \end{center}
  \end{figure}
\end{frame}

\begin{frame}
  \frametitle{Comparaison - distances intra-cluster}
  \begin{figure}[ht]
    \begin{center}
      \includegraphics[width=75mm]{../clustering/teddy_toy_data_0_clustering_sumdst_10}
      \caption{Somme des distances intra-cluster pour le dataset teddy-toy.}
      \label{fig:teddy_toy_data_04}
    \end{center}
  \end{figure}
\end{frame}

\begin{frame}
  \frametitle{Comparaison - nombre d'it�ration}
  \begin{figure}[ht]
    \begin{center}
      \includegraphics[width=75mm]{../clustering/teddy_toy_data_0_clustering_sumit_10}
      \caption{Nombre d'it�rations pour le dataset teddy-toy.}
      \label{fig:teddy_toy_data_05}
    \end{center}
  \end{figure}
\end{frame}

\section*{Questions}
\begin{frame}
  \frametitle{Questions}
\end{frame}

\end{document}
